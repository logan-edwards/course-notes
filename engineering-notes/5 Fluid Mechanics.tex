\section{Fluid Mechanics}

\begin{center}
    Notes from \textbf{MECHENG 320: Introduction to Fluid Mechanics}

    Taken at University of Michigan, Fall 2025

    using \textit{Fundamentals of Fluid Mechanics}, by Munson, Okiishi, and Young
\end{center}

\subsection{Fluid Properties}

A \textbf{fluid} is a substance which deforms \textit{continuously} when acted upon by a shearing force of any magnitude. As opposed to a solid, which deforms to a finite extent when a force is applied, a fluid is easily deformed and this deformation does not stop; in other words, a fluid can be made to \textit{flow}. The study of fluids tends to involve a \textbf{continuum approximation}, where the quantities associated with the fluid vary continuously, as opposed to an atomistic model.

%%% Viscosity

\textbf{Viscosity} is a property of a fluid which measures its ability to resist flow. \textbf{Dynamic viscosity} is the proportionality constant relating fluid shearing stress and the rate of shearing strain, \[\tau = \mu \frac{du}{dy}\] Note that $\frac{du}{dy}$ is speed per distance, but can be shown to be equivalent to the rate of shearing strain. Dynamic viscosity has SI units $\frac{\text{kg}}{{m}\cdot\text{s}}$. \textbf{Kinematic viscosity} is found by normalizing dynamic viscosity to the density of a fluid, given by \[\nu = \frac{\mu}{\rho}\] Viscosity is dependent on the temperature of a fluid. If $\mu$ is linear, i.e. $\tau / (du/dy)$ is constant, then a fluid is called \textit{Newtonian}; if this relationship is nonlinear, a fluid is called \textit{Non-Newtonian}.

\textbf{Pressure} is another property of a fluid. Pressure is a measure of the normal stress in a fluid. Pressure is a \textit{scalar} quantity, and forces arising due to pressure act \textit{normal} to the interaction surface. The \textbf{compressibility} of a fluid is given by the \textit{bulk modulus}, which quantifies the pressure required to compress a fluid: \[E_v = -\frac{dp}{dV/V} = \frac{dp}{d\rho/\rho}\] As $E_v\to\infty$, the quantity $\frac{d\rho}{\rho}\to 0$, which is the \textit{incompressibility} condition. As opposed to viscosity, bulk modulus is a property of the fluid. Bulk modulus is related to speed of sound in a medium by \[c = \sqrt{\frac{{E_v}}{\rho}}\]

%%% Read through textbook for special cases, i.e. with gases

\textbf{Surface tension} occurs when there is a pressure difference between the fluid and surrounding medium. Surface tension is denoted by $\sigma$, and is a force per length. In general, for a length $L$ of an interface (i.e. $2\pi R$ for the arclength of a circular droplet) and an area enclosed by the interface $A$ (i.e. $\pi R^2$ in this case), the pressure difference arising within a droplet from surface tension is given by force balance: \[\sigma L = \Delta p A \text{; or, for circular droplets, } \sigma \cdot 2\pi R = \Delta p \cdot \pi R^2\]

\newpage

\subsection{Fluid Statics} % chapter 2.1-2.8

\textbf{Fluid statics}, sometimes called \textit{hydrostatics}, is a subset of fluid mechanics studying forces acting on and caused by fluids at rest, and therefore without shearing stresses. In absence of shearing stress, pressure acts equally in each direction and therefore can be treated as a scalar.

\textbf{Pascal's Law}. Pressure at a point in a fluid at rest, or in motion, is independent of direction \textit{so long as there are no shearing stresses present}. %%% this might be unimportant

% Hydrostatic Equation

In a static fluid, pressure distribution can be shown to be given by:
\[\frac{\partial p}{\partial x} = \frac{\partial p}{\partial y} = 0,\ \frac{\partial p}{\partial z} = -\rho g\] In other words, at a given depth $z$, movement in the $x$ or $y$ direction (i.e. movement at the same depth) does not affect pressure. Pressure can only change by a change in depth $z$. Given constant $\rho g$, as would occur in an incompressible fluid where $\rho(z)$ is constant, integration yields the \textbf{hydrostatic equation}: \[\frac{\partial p}{\partial z} = -\rho g \implies \boxed{p = p_\text{atm} + \rho g h}\hspace{0.5in}\text{(Hydrostatic Equation)}\] In the hydrostatic equation, $p_\text{atm}$ is the atmospheric pressure at the fluid surface, and $h$ is the depth of immersion. Therefore, static pressure is independent of vessel shape and only depends on depth. Pressure may be measured as an \textit{absolute pressure}, measured with respect to a perfect vacuum (i.e. from 0), or as a \textit{gage pressure}, measured with respect to the local atmospheric pressure (i.e. $p_\text{gage} = p - p_\text{atm}$. The hydrostatic equation gives an \textit{absolute pressure}.

\begin{shaded}
    \textbf{Force on Submerged Objects}. Hydrostatics enables the study of forces acting on submerged objects. Assuming a fluid is open (i.e. at $h=0$ the fluid has pressure $P_\text{atm}$), the resultant forces due to the fluid may be calculated using a static balance. There are two cases in general: either a planar (flat) surface, or a curved surface, oriented to the free surface.
    
    \textbf{Flat Surface}. Let $\theta$ be the angle between the surface and the free surface. From the hydrostatic equation, it can be shown that the resultant force has magnitude \[F_R = \rho g h_c A\] where $h_c$ is the height of the centroid and $A$ is the area of the shape. The location $(x_R, y_R)$ of the resultant force is the \textbf{center of pressure}, which is given by \[x_R = \frac{I_{xy,c}}{Ay_c} + x_c\hspace{0.5in} y_R = \frac{I_{x,c}}{y_cA} + y_c\] where $I_{x,c}$ is the second moment of area with respect to the centroid, and $I_{xy,c}$ is the \textit{product of interia} with respect to the centroid. Note that the center of pressure is always below the centroid for any submerged shape.

    \textbf{Curved Surface}. In general, net force acting on a submerged body may be found by balancing forces on a fluid volume. In particular, the horizontal forces on a volume must balance and the vertical forces on a volume must balance, as shown below:
    \begin{center}
        \includegraphics[scale=0.35]{Images/fluids_submergedbalance.png}
    \end{center}
    The procedure for computing forces on a general submerged body is as follows:
    \begin{enumerate}
        \item Compute $F_H = F_2$. The force acting horizontally on the curved surface must equal the force exerted on the flat, vertical plane through the fluid. Thus, $F_H$ is given by \[F_H = \rho g h_\text{c,plane} A_\text{plane}\] where $h_\text{c,plane}$ is the centroid (center) of the plane, i.e. half the depth of the body, and $A_\text{plane}$ is the area of the flat plane.
        \item Compute $F_V = W + F_1$. If the surface of the fluid is free, $F_1=0$. Weight is computed by \[F_V = \rho g V_\text{fluid}\]
        \item Compute reaction force. The magnitude of the force is given by \(F = \sqrt{F_H^2 + F_V^2}\) and the force acts in the unit direction $\hat{e} = \left(\frac{F_H}{F},\ \frac{F_V}{F}\right)$ from the origin.
    \end{enumerate}
\end{shaded}

\textbf{Buoyancy} is the net vertical force acting on a submerged object due to fluid pressure. Buoyancy force $F_b$ is given by \[F_b = \rho_\text{fluid}gV_\text{displaced}\] \textbf{Archimedes Principle} gives the condition for floating (where $W= \rho_\text{solid}gV$):

\begin{enumerate}
    \item[] If $F_b > W$, the body will \textit{rise} in the fluid.
    \item[] If $F_b = W$, the body is \textit{neutrally buoyant} and will not rise or sink in the fluid.
    \item[] If $F_b < W$, the body will \textit{sink} in the fluid.
\end{enumerate}

A floating body is \textit{stable} if the buoyancy force, acting at the center of buoyancy, produces a moment which restores the body to its upright position, a \textit{restoring couple}. A body which produces an \textit{overturning couple} is \textit{unstable}. The \textit{metacenter} is the intersection point between the line through the original center of buoyancy and the new center of buoyancy. If the metacenter is above the body, the body is stable; if the metacenter is on the body, the body is unstable.

\subsection{Fluid Kinematics}

% Lagrangian vs Eulerian decsriptions
% material derivatives

Fluid kinematics is the description of fluid motion without regard to forces acting on and caused by the fluid. There are two primary kinematic frameworks for analyzing fluids:

\begin{enumerate}
    \item[] \textbf{Eulerian Description of Fluid Kinematics}. Flow is observed with respect to a fixed position as time progresses. This framework provides a \textit{lab-fixed reference frame}.
    \item[] \textbf{Lagrangian Description of Fluid Kinematics}. Individual particles of fluid are tracked as they flow. This model is convenient for kinematics but will need to be modified when dealing with forces (i.e. fluid dynamics).
\end{enumerate}

For either framework, the assumption is made that for any fluid particle (i.e. a small collection of fluid), mass is constant but shape and volume may change (depending on the compressibility of the fluid). Fluid velocity constitutes a vector field, where $\vec{v}(x,y,z,t) = u(x,y,z,t)\hat{i} + v(x,y,z,t)\hat{j}+w(x,y,z,t)\hat{k}$ denotes the velocity at any point $(x,y,z)$ and any time $t$. Familiarly, if $\vec{r}$ is the position vector from the origin to a particle of fluid, \[\vec{v} = \frac{d\vec{r}}{dt} = \vec{v}(\vec{r}(t),t) = \vec{v}(x,y,z,t)\] As expected, the acceleration of a fluid is given by differentiation of $\vec{v}$:
\begin{eqnarray*}
    \vec{a}(t) &=& \frac{d\vec{v}}{dt}\\
    &=& \frac{\partial \vec{v}}{dt} + \frac{\partial \vec{v}}{\partial x} \cdot \frac{\partial x}{dt} + \frac{\partial \vec{v}}{\partial y} \cdot \frac{\partial y}{dt} + \frac{\partial \vec{v}}{\partial z} \cdot \frac{\partial z}{dt}\\
    &=& \frac{\partial \vec{v}}{\partial t} + u\frac{\partial \vec{v}}{\partial x} + v\frac{\partial \vec{v}}{\partial y} + w\frac{\partial \vec{v}}{\partial z}
\end{eqnarray*} Then, for $\vec{v}=u\hat{i} + v\hat{j} + w\hat{k}$, the components of acceleration are given by: \[\begin{cases}
    a_x &= \displaystyle\frac{\partial u}{\partial t} + u\frac{\partial u}{\partial x} + v\frac{\partial u}{\partial y} + w\frac{\partial u}{\partial z}\\
    \\
    a_y &= \displaystyle\frac{\partial v}{\partial t} + u\frac{\partial v}{\partial x} + v\frac{\partial v}{\partial y} + w\frac{\partial v}{\partial z}\\
    \\
    a_z &= \displaystyle\frac{\partial w}{\partial t} + u\frac{\partial w}{\partial x} + v\frac{\partial w}{\partial y} + w\frac{\partial w}{\partial z}
\end{cases}\] Acceleration is an example of a \textit{material derivative}, also called a \textit{substantial derivative}. The material derivative is an operator defined by: \[\frac{D}{Dt}(-) = \frac{\partial}{\partial t} (-) + \vec{v}\cdot \nabla (-)\] which is essentially the partial derivative of a property with respect to time plus the dot product of velocity with the gradient of the property. Physically the material derivative gives the total change in a property of a fluid flow as an observer moves with the flow. This is composed first of local temporal effects on the property, and the spatial \textit{convective derivative}, which gives the rate of change of the property as fluid moves through locations. In steady-state flow, the temporal term is zero, but the total derivative may still be nonzero depending on the convective terms.

There are several ways to visualize fluid flow:
\begin{enumerate}
    \item \textbf{Streamlines}. If a vector field is known or approximated analytically, the \textit{streamlines} may be drawn across the flow field. Streamlines are drawn by definition as a line instantaneously tangent to the flow field. From this definition, the slope of a streamline in two dimensions is given by \[\frac{dy}{dx} = \frac{v}{u}\] where the slope of the streamline satisfies this differential equation, and the solution given by separation of variables is the slope of the streamline as a function of position.
    \item \textbf{Pathlines} (Lagrangian concept). A given volume of fluid is somehow marked (i.e. by dye) and tracked. In a lab, this can be used to plot position of a particular particle as it progresses through time. Position is given by familiar kinematic relationships. \[\frac{dx}{dt} = u,\ \ \frac{dy}{dt}=v\]
    \item \textbf{Streaklines}. A dye or similarly visible material is inserted into a flow. The streaks as the dye moves are visible and allow for flow visualization. This method is common in laboratories due to its simplicity.
\end{enumerate}
In steady flow, these lines are all equivalent; in this case, these lines are typically simply called the \textit{streamlines} of the flow field.

\newpage

\subsection{Fluid Dynamics}

Fluid dynamics is the study of forces causing and resulting from fluid motion. By applying Newton's second law to a fluid particle \textit{moving along its streamline} and under a set of simplifying assumptions, Bernoulli's equation can be derived. Bernoulli's' equation essentially states conservation of energy along a streamline: \begin{equation*}P + \frac{1}{2}\rho v^2 + \rho gh = \text{constant}\tag{\text{Bernoulli's Equation}}\end{equation*} $P$ is the \textit{static pressure}, where $\frac{1}{2}\rho v^2$ is the \textit{dynamic pressure} which is the product of velocity (in magnitude) and density, and $\rho g h$ is the familiar hydrostatic pressure. Applied at two points along a streamline, Bernoulli's Equation is written as follows:
\[
    P_1 + \frac{1}{2}\rho v_1^2 + \rho gh_1 = P_2 + \frac{1}{2}\rho v_2^2 + \rho g h_2
\]

Bernoulli's Equation is typically written in one of three forms:
\begin{shaded}
    \textbf{Bernoulli's Equation}. Bernoulli's equation can be rearranged in several ways to solve for different quantities:

    \begin{enumerate}
        \item \textbf{Pressure Form}. Units are pressure. \[P + \rho \frac{v^2}{2} + \rho gh = \text{const.}\]
        \item \textbf{Energy Form}. Units are energy per mass. \[\frac{P}{\rho} + \frac{v^2}{2} + gh = \text{const.}\]
        \item \textbf{Head Form}. Units are length. \[\frac{P}{\rho g} + \frac{v^2}{2g} + h = \text{const.}\] In the head form, the quantity $P/(\rho g)$ is referred to as the \textit{pressure head}, and $h$ as the \textit{head length}.
    \end{enumerate}
\end{shaded}

It should be noted that Bernoulli's equation makes several simplifying assumptions and is therefore only valid for very specific fluid flows. In particular, Bernoulli's equation is valid only under \textit{steady-state}, \textit{inviscid} flow, with a \textit{constant density} and evaluation occurs between two points \textit{along the same streamline}.

\begin{framed}
\textbf{Example}. \textit{Stagnation Pressure}. Stagnation pressure is the pressure resulting from a fluid coming to a complete stop at a point called a \textit{stagnation point}, i.e. kinetic energy is converted completely into pressure. This pressure is given by: \begin{align*}
    P_1+\frac{1}{2}\rho v_1^2 = &P_2 + \cancelto{0}{\frac{1}{2}\rho v_2^2}\\
    \implies &P_2 = P_1 + \frac{1}{2}\rho v_1^2
\end{align*}
\end{framed}

Bernoulli's equation can be modified to relax the assumption of steady-state. The result is the \textbf{unsteady Bernoulli equation}, which requires knowledge of the velocity at every point along a streamline. \[\underbrace{p_1+\frac{1}{2}\rho v_1^2 + \rho g z_1 = p_2+\frac{1}{2}\rho v_2^2 + \rho g z_2}_\text{Bernoulli's equation} + \underbrace{\rho \int_{s_1}^{s_2} \frac{\partial v}{\partial t}ds}_\text{Unsteady effects}\] $s_1$ and $s_2$ are the initial and final points considered along the same streamline.

\newpage 

The description of a fluid given by Bernoulli's equation is limited. More generally, there is an interest in applying to laws of physics directly to the fluid. A \textit{control volume} allows these laws to be applied, and the fluid may then be described in terms of integral equations or differential equations. In particular, the following relationships are of interest, and can be obtained through integral or differential analysis on the control volume: \begin{enumerate}
    \item Conservation of Mass.
    \item Conservation of Momentum.
    \item Conservation of Energy.
\end{enumerate}
The coupled system given by (1) and (2) will be the \textit{Navier-Stokes Equations}.

\begin{shaded} \textbf{Control Volume Analysis}. A control volume may be defined in either a Lagrangian or Eulerian framework. In a Lagrangian framework, the control volume is a collection of fluid particles; this volume moves with the flow and always contains the same particles. In an Eulerian framework, the control volume is an arbitrary region in space through which fluid may flow; the matter in this volume changes with time.

Suppose $B$ is any extensive property of a system (momentum, energy, force, etc). Then, define $b=B/m$, where $m$ is mass and $b$ is therefore an intensive property. Reynold's Transport Theorem relates the time derivatives of the intensive properties of a system to a control volume.

\textbf{Reynold's Transport Theorem}. The rate of change of an extensive property is equal to the combination of the rate of change of the extensive property inside the control volume $CV$ and the rate at which the extensive property is flowing across the surface of the control volume, the control surface $CS$. \[\frac{d}{dt} B_\text{system} = \frac{\partial}{\partial t} \int_{\text{CV}} \rho b dV + \int_{CS} \rho b \vec{w}\cdot \hat n dA\] The quantity $\vec{w}:=\vec{v}-\vec{v}_{CV}$ is the difference between true velocity and surface velocity.
\end{shaded}

\textbf{Derivation of the Continuity Equation}. Applying Reynold's transport theorem to conservation of mass immediately gives the \textit{continuity equation} (with $B = m \implies b=m/m=1$): \[\frac{dm}{dt} = 0 = \frac{\partial}{\partial t}\int_{CV} \rho dV + \int_{CS} \rho \vec{w}\cdot\hat{n} dA\tag{Continuity Equation}\] Note that a positive flux term implies outflow, and a negative flux term implies inflow. The continuity equation can be simplified in several ways under different assumption sets. For \textit{incompressible flow}, density is constant, so the continuity equation becomes: \[\frac{\partial}{\partial t}\int_{CV} dV + \int_{CS}\vec{w}\cdot \hat{n} dA = 0\] which is a statement on conservation of volume. For steady flow with constant control volume, the temporal variation disappears and $\vec{w}=\vec{v}$, the velocity of fluid flow. Thus, this equation reduces to: \[\int_{CS} \vec{v}\cdot \hat{n}dA = 0 \iff \sum \dot m_\text{in} = \sum \dot m_\text{out}\] Finally, if $\rho\vec{w}$ is a constant term over the entire surface, and if flow is steady, the flux term becomes: \[\int_{CS} \rho \vec{w}\cdot \hat{n} dA = \sum_{i} \rho \vec{w}\cdot \hat{n} dA_i\]

\textbf{Derivation of Conservation of Linear Momentum}

Newton's 2nd law, in most general momentum form, is given by \[\frac{d}{dt} \left(m\vec{v}\right)_\text{sys} = \sum \vec{F}_\text{sys}\] Observe that, if a control volume $CV$ is coincident with the system, then $\sum \vec{F}_\text{sys} = \sum \vec{F}_{CV}$. Thus, Reynold's transport theorem may be applied: \begin{eqnarray*}
    \sum \vec{F}_\text{sys} &=& \frac{d}{dt}(m\vec{v})_\text{sys}\\ &=& \underbrace{\frac{\partial}{\partial t} \int_\text{CV} \rho \vec{v} dV}_\text{change inside volume} + \underbrace{\int_{CS} \rho \vec{v} \left[(\vec{v}-\vec{v}_{CV}) \cdot \hat{n}\right] dA}_\text{flux through surface}\\
\end{eqnarray*} In general, three kinds of forces act on fluid in a control volume: \begin{enumerate}
    \item[] \textbf{Surface forces}. Surface forces act on the surface and are proportional to the surface area, such as pressure forces.
    \item[] \textbf{Body Forces}. Body forces act on the entire volume and are proportional to volume, such as gravity.
    \item[] \textbf{External Forces}. Force resulting from bodies within the control volume. These forces are unknown; common examples include lift and drag of a body within a control volume.
\end{enumerate} Therefore, a complete force balance is given by: \[\frac{\partial}{\partial t}\int_{CV} \rho \vec{v} dV + \int_{CS} \rho \vec{v}[(\vec{v}-\vec{v}_{CV})\cdot \hat{n}]dA = \underbrace{-\int P \hat{n} dA}_\text{surface forces} + \underbrace{\int_{CV}\rho \vec{g} dV}_\text{body forces} + \vec{F}_\text{viscous} +  \vec{F}_\text{ext}\] where $\vec{F}_\text{ext}$ are external forces and $\vec{F}_\text{viscous}$ are the viscous forces, which can be ignored if a fluid is approximately inviscid.

\textbf{Derivation of Conservation of Energy}. Let $e=E/m$, or energy per unit mass. Note that $e=\hat{u} + \frac{1}{2}v^2+gz$, for internal energy, kinetic energy, and potential energy respectively. Then, applying Reynold's transport theorem to the first law of thermodynamics gives conservation of energy for a fluid: \begin{eqnarray*}
    \dot Q_\text{net,sys}+\dot W_\text{net,sys} &=& \frac{d}{dt}E_\text{sys}\\
    &=& \frac{\partial}{\partial t}\int_{CV} \rho e dV + \int_{CS} \rho e (\vec{v}-\vec{v}_{CV})\cdot \hat{n}dA
\end{eqnarray*} Note that this is a scalar equation. Note that in an \textit{adiabatic} process, $\dot Q_\text{net,in} = 0$. In general, $\dot Q_\text{net,in} = \dot Q_{in}-\dot Q_{out}$. Similarly, $\dot W_\text{net,in} = \dot W_\text{in} - \dot W_\text{out}$, where work may be done by any of the following mechanisms: \begin{enumerate}
    \item[-] $\dot W_\text{pressure,in}$, work done by pressure forces on control surface.
    \item[-] $\dot W_\text{viscous}$, work done on the control surface by viscous forces, which may often be neglected.
    \item[-] $\dot W_\text{shaft,in}$, which is work done by a machine on the fluid in the control volume (i.e. pumps, pistons, fans, etc).
\end{enumerate} Neglecting viscous forces, a complete energy balance is given by: \[\frac{\partial}{\partial t}\int_{CV} \rho e dV + \int_{CS} \rho e (\vec{v}-\vec{v}_{CV})\cdot \hat{n} dA = \dot Q_\text{net,in} + \dot W_\text{shaft,in} \underbrace{-\int P(\vec{v}-\vec{v}_{CV})\cdot \hat{n} dA}_\text{work by pressure forces}\] The energy balance may be simplified under certain assumptions to produce the \textit{mechanical energy equation}, which is also known as the \textit{extended Bernoulli equation}. In particular, assuming steady flow, incompressibility, and a \textit{single inlet and single outlet with uniform flow}, integration over the control volume and algebraic rearrangement gives the following expression: \[\frac{P_\text{out}}{\rho}+\frac{1}{2}v_\text{out}^2 + gz_\text{out} = \frac{P_\text{in}}{\rho} + \frac{1}{2}v_\text{in}^2 + gz_\text{in} + w_\text{shaft,in} - \underbrace{(\hat{u}_\text{out}-\hat{u}_\text{in}-q_\text{net,in})}_\text{losses}\tag{Extended Bernoulli Equation}\] where that $q_\text{net,in}$ is defined by $\dot Q/\dot m$, and $w_\text{shaft,in}$ is defined by $\dot W/\dot m$. Note that this equation reduces to the Bernoulli equation if there is no shaft work or energy losses.

%%% Read Section 6.1

The integral form can be reduced to differential form. In particular, starting with the integral form of the conservation of mass, one can arrive at the \textbf{continuity equation}: \begin{eqnarray*}
    0 &=& \frac{\partial}{\partial t}\int_{CV} \rho dV + \int_{CS}\rho(\vec{v}-\vec{v}_{CV})\cdot\hat{n}dA\\
    \text{assuming constant, fixed CV: }\hspace{0.25in}&=&  \int_{CV}\frac{\partial\rho}{\partial t}dV + \int_{CS} \rho \vec{v}\cdot\hat{n}dA \hspace{0.25in}\\
    \text{applying divergence theorem:\hspace{0.25in} }&=& \int_CV \frac{\partial \rho}{\partial t}dV + \int_{CV} \nabla\cdot (\rho \vec{v})dV\hspace{0.25in}\\
    &=& \int_{CV} \left[\frac{\partial\rho}{\partial t}+ \nabla\cdot (\rho \vec{v})\right]dV\\
    \text{ as $V\to 0$, i.e. at a point:\hspace{0.25in}}&=& \frac{\partial\rho}{\partial t} + \nabla \cdot (\rho \vec{v})
\end{eqnarray*} Thus, the \textbf{continuity equation}, which gives a condition which \textit{every point} must satisfy to satisfy conservation of mass, is given by the differential equation: \[\frac{\partial \rho}{\partial t} + \nabla \cdot (\rho\vec{v}) = 0\tag{Continuity Equation}\] Assuming incompressibility, all derivatives involving $\rho$ are zero; then, the \textit{incompressible} continuity equation is given by: \[\nabla\cdot \vec{v} =0\tag{Incompressible Continuity}\] Then, the continuity equation will hold on a fluid field \textit{if and only if} the fluid is incompressible on that field. Note that a flow may be incompressible and still have variable density; in particular, an incompressible fluid may have variable density layers (i.e. water which is heated from the bottom is incompressible in the liquid phase, but the density increases further from the heat source).

\begin{shaded}
    \textbf{Stream Functions}. Consider the simple case of steady, constant density, two-dimensional flow. Define $\psi(x,y)$ such that \[u = \frac{\partial \psi}{\partial y},\hspace{0.5in}v=-\frac{\partial\psi}{\partial x}\] Then, $\psi(x,y)$ allows a flow field to be defined with a single function, instead of two velocity functions (i.e. $(u(x,y),v(x,y))$. Note that $\psi$ plus any constant defines the same flow field; by convention, $\psi(x,y)$ is typically written without any such constant. $\psi(x,y)$ has several useful properties:
    
    \textbf{The velocity field defined by $\psi$ is incompressible}. Applying the continuity equation, \[\frac{\partial u}{\partial x} + \frac{\partial v}{\partial y} = 0 \implies \underbrace{\frac{\partial^2\psi}{\partial x\partial y} -\frac{\partial^2\psi}{\partial x\partial y}}_{\text{definition of }\psi}= 0\]
    
    \textbf{Lines of constant $\psi$ are streamlines}. Suppose $\psi(x,y)=c$. Then, \[\underbrace{0 = d\psi}_\text{on const. $\psi$} = \frac{\partial\psi}{\partial x} dx + \frac{\partial \psi}{\partial y}dy = -vdx + udy \implies \underbrace{\frac{dy}{dx}=\frac{v}{u}}_\text{streamline def'n}\]

    \textbf{The volume flow rate between streamlines is the difference between streamlines}. Specifically, if $c_1$ and $c_2$ are two streamlines given by $\psi$ with $c_2>c_1$, then $q=c_2-c_1$ is the volumetric flow rate \textit{per unit depth}. 
\end{shaded}

\begin{framed}
\textbf{Example}. \textit{Derive the stream function $\psi(x,y)$ for a flow field given by $(u,v) = (Ax,-Ay)$.}

There are two constrains $\psi$ must satisfy: \[\frac{\partial\psi}{\partial y} = u,\hspace{0.5in}\frac{\partial \psi}{\partial x} = -v\] Integrating $u$ gives: \[\frac{\partial\psi}{\partial y}dy = Axdy\implies d\psi = Axdy\implies \psi(x,y) = Axy+f(y)\] Then, integrating $v$ gives: \[\frac{\partial\psi}{\partial x}dx = Aydx \implies d\psi = Aydx \implies \psi(x,y) = Ayx+g(x)\] Then, it must be the case that $g(x)=f(y)$; this can only be the case if $g(x)=f(y)=c$ for some constant $c$. This constant term may be neglected (by convention) to give: \[\boxed{\psi(x,y) = Axy}\] This stream function defines the given flow field. Note that the streamlines are given by \[\psi(x,y) = k \implies y = \frac{k}{Ax}\] where $k$ is any arbitrary constant. For any $k$, this equation defines a unique streamline.
\end{framed}

The integral form of the conservation of momentum can also be reduced to differential form by a similar process. The final result is the following equation:
\[\frac{D\vec{u}}{Dt} = -\frac{1}{\rho}\nabla P + \underbrace{\nu\nabla^2\vec{u}}_{\nu=\mu/\rho} + \vec{g}\tag{Conservation of Momentum}\] Written in terms of its components, this is equivalently given by: 
\begin{align*}
    \frac{\partial u}{\partial t} + u \frac{\partial u}{\partial x} + v\frac{\partial u}{\partial y} + w\frac{\partial u}{\partial z} &= -\frac{1}{\rho}\frac{\partial p}{\partial x} + \nu\left(\frac{\partial^2 u}{\partial x^2} + \frac{\partial^2 u}{\partial y^2} + \frac{\partial^2 u}{\partial z^2}\right) + g_x\tag{$x$}\\
    \\
    \frac{\partial v}{\partial t} + u \frac{\partial v}{\partial x} + v\frac{\partial v}{\partial y} + w\frac{\partial v}{\partial z} &= -\frac{1}{\rho}\frac{\partial p}{\partial y} + \nu\left(\frac{\partial^2 v}{\partial x^2} + \frac{\partial^2 v}{\partial y^2} + \frac{\partial^2 v}{\partial z^2}\right) + g_y\tag{$y$}\\
    \\
    \frac{\partial w}{\partial t} + u \frac{\partial w}{\partial x} + v\frac{\partial w}{\partial y} + w\frac{\partial w}{\partial z} &= -\frac{1}{\rho}\frac{\partial p}{\partial z} + \nu\left(\frac{\partial^2 w}{\partial x^2} + \frac{\partial^2 w}{\partial y^2} + \frac{\partial^2 w}{\partial z^2}\right) + g_z\tag{$z$}
\end{align*}

The continuity equation and the conservation of momentum, when taken together, constitute the \textbf{Navier-Stokes equations}.

For many problems, it is convenient to write the Navier-Stokes equations in terms of cylindrical coordinates. Therefore, the Navier-Stokes equations are equivalently written as:
\begin{align*}
    \frac{1}{r}\frac{\partial}{\partial r}(rv_r) + \frac{1}{r}\frac{\partial v_\theta}{\partial\theta} + \frac{\partial v_z}{\partial z} &= 0\tag{Continuity}\\
    \\
    \rho\left(\frac{\partial v_r}{\partial t} + v_r \frac{\partial v_r}{\partial r} + \frac{v_\theta}{r}\frac{\partial v_r}{\partial \theta} - \frac{v_\theta^2}{r} + v_z \frac{\partial v_r}{\partial z}\right) &= -\frac{\partial p}{\partial r}+\mu \left[\frac{1}{r}\frac{\partial}{\partial r} \left(r \frac{\partial v_r}{\partial r}\right) - \frac{v_r}{r^2} + \frac{1}{r^2} \frac{\partial^2 v_r}{\partial \theta^2} - \frac{2}{r^2}\frac{\partial v_\theta}{\partial \theta} + \frac{\partial^2 v_r}{\partial z^2}\right] + \rho g_r\tag{$r$}\\
    \\
    \rho\left(\frac{\partial v_\theta}{\partial t} 
    + v_r \frac{\partial v_\theta}{\partial r} 
    + \frac{v_\theta}{r}\frac{\partial v_\theta}{\partial \theta} 
    + \frac{v_\theta v_r}{r} + v_z \frac{\partial v_\theta}{\partial z}\right) 
    &= -\frac{1}{r}\frac{\partial p}{\partial\theta} + \mu \left[\frac{1}{r}\frac{\partial}{\partial r} \left(r \frac{\partial v_\theta}{\partial r}\right) 
    - \frac{v_\theta}{r^2} 
    + \frac{1}{r^2} \frac{\partial^2 v_\theta}{\partial \theta^2} 
    + \frac{2}{r^2}\frac{\partial v_r}{\partial \theta} 
    + \frac{\partial^2 v_\theta}{\partial z^2}\right] + \rho g_\theta\tag{$\theta$}\\
    \\
        \rho\left(\frac{\partial v_z}{\partial t} + v_r\frac{\partial v_z}{\partial r} + \frac{v_\theta}{r}\frac{\partial v_z}{\partial \theta} + v_z\frac{\partial v_z}{\partial \theta}\right) &= -\frac{\partial p}{\partial z} + \mu\left[\frac{1}{r}\frac{\partial}{\partial r}\left(r\frac{\partial v_z}{\partial r}\right) + \frac{1}{r^2}\frac{\partial^2 v_z}{\partial\theta^2} + \frac{\partial^2 v_z}{\partial z^2}\right] + \rho g_z\tag{$z$}
\end{align*}

\begin{framed}
    \textbf{Example: Plane Poiseuille Flow}. Find an expression for the velocity of a fluid between two fixed, infinitely long plates separated by a distance $b$. Assume the flow is driven by a known pressure differential $\frac{\partial p}{\partial x}$, and that the fluid is incompressible.

    \textit{Assumptions}.
    \vspace{-1.5em}
    \begin{enumerate}
        \item Planar flow$\implies$ no flow in $z$ direction.
        \item Infinitely long plates$\implies$ flow is fully developed$\implies$no change in $x$ direction, i.e. $\frac{\partial}{\partial x}\to 0$.
        \item Incompressible fluid$\implies\nabla\cdot \vec{v}=0$
        \item Steady state$\implies$no change with time$\implies \frac{\partial}{\partial t}=0$
        \item Neglect gravity (planar orientation not specified).
    \end{enumerate}

    \textit{Boundary Conditions}.
    \vspace{-1.5em}
    \begin{enumerate}
        \item No-slip boundary condition at the walls: $u(0)=u(b)=0$.
        \item Impermeable walls: $v(0)=v(b)=0$.
    \end{enumerate}

    Since the fluid is incompressible, $\nabla\cdot \vec{v}=0$: \[\nabla\cdot\vec{v} = \cancelto{0,\text{fully developed}}{\frac{\partial u}{\partial x}} + \frac{\partial v}{\partial y} = 0 \implies \frac{\partial v}{\partial y}=0 \implies v = v_0\] Applying boundary conditions, $v(0)= 0$, so $v_0=0$. Therefore, $v=0$. Now, considering the momentum balance in the $x$ direction: \[\cancelto{(1)}{\frac{\partial u}{\partial t}} + \cancelto{(2)}{u\frac{\partial u}{\partial x}} + \cancelto{(3)}{v\frac{\partial u}{\partial y}} = -\frac{1}{\rho}\frac{\partial p}{\partial x} + \nu\left(\cancelto{(4)}{\frac{\partial^2 u}{\partial x^2}} + \frac{\partial^2u}{\partial y^2}\right)\] These terms are zero because the flow is assumed to be steady state (1), and fully developed (2,4). From the continuity equation, $v=0$ (2). Therefore, the expression becomes: \[\frac{1}{\rho}\frac{\partial p}{\partial x} = \nu\frac{\partial^2 u}{\partial y^2}\] Now, it must be checked whether $p$ is a function of $y$; if not, this expression can be integrated with $\frac{\partial p}{\partial x}$ treated as a constant. Consider momentum in the $y$ direction: \[\cancel{\frac{\partial v}{\partial t}} + \cancel{u\frac{\partial v}{\partial x}} + \cancel{v\frac{\partial v}{\partial y}} = -\frac{1}{\rho}\frac{\partial p}{\partial y} + \nu\left(\cancel{\frac{\partial^2 v}{\partial x^2}} + \cancel{\frac{\partial^2v}{\partial y^2}}\right)\] $v$ is constant, so all derivatives are $0$. Therefore, $\frac{\partial p}{\partial y} = 0$ meaning $p$ does not vary with $y$, i.e. $p$ is not a function of $y$. Therefore, the original equation can be safely integrated: \[\frac{1}{\rho}\frac{\partial p}{\partial x} = \nu \frac{\partial^2 u}{\partial y^2} \implies \frac{1}{\nu\rho}\frac{\partial p}{\partial x}y + c_1 = \frac{\partial u}{\partial y}\implies\frac{1}{2\nu\rho}\frac{\partial p}{\partial x}y^2 + c_1 y + c_2 = u(y)\] Applying boundary conditions: \[u(0) = 0 \implies c_2 = 0;\hspace{0.1in} u(b)=0\implies \frac{1}{2\nu\rho}\frac{\partial p}{\partial x}b^2 + c_1b = 0 \implies c_1 = -\frac{b}{2\nu\rho}\frac{\partial p}{\partial x}\] Substituting $c_1$ and $\mu = \nu\rho$ gives the (quadratic) velocity profile: \[\boxed{u(y) = \frac{1}{2\mu}\frac{\partial p}{\partial x}\left(y^2 - yb\right)}\]
\end{framed}

\subsection{Dimensional Analysis}

The basic principle of \textit{dimensional analysis} is that all physical laws must be independent of any particular system of measurement; in other words, any physical law can be expressed equivalently in any self-consistent unit system. Dimensional analysis involves formulating problems in terms of dimensionless variables and parameters.

The benefit of dimensional analysis is that it substantially reduces the number of necessary experiments to determine a relationship between parameters. In particular, if a physical phenomenon depends on $n$ variables, and $k$ data points are needed for each experiment, then $n^k$ experiments must be conducted; dimensional analysis often reduces this into fundamental, non-dimensional parameters which define.

\begin{shaded}
\textbf{Buckingham Pi Theorem}. Let $q=(q_1,\dots,q_n)$ be $n$ variables such that there exists a functional relation of the form $f(q)=0$. Then, the $n$ variables can be combined to form exactly $m=n-r$ independent, non-dimensional variables $\Pi=(\Pi_1,\dots,\Pi_m)$, where $r$ is the number of independent dimensions, which is the rank of the dimensional matrix. Equivalently, $f(q)$ can be written in the form $\phi(\Pi)=0$.
\end{shaded}

The general steps for this dimensional analysis are as follows:
\begin{enumerate}
    \item[1.] Determine relevant variables and parameters $x_1,\dots, x_n$.
    \item[2.] Form the \textit{dimensional matrix}.
    \item[3.] Determine the rank $r$ of the dimensional matrix.
    \item[4.] Determine the number of dimensionless groups $m = n - r$.
    \item[5.] Construct the dimensionless $\Pi$-groups. That is, each group is of the form $\Pi_i = x_i x_{r_1}^{a_1} \dots x_{r_r}^{a_r}$ where $x_{r_i}$ are the repeated parameters which are the same for each $\Pi$ group. The exponents $a_1,\dots a_r$ are chosen so that $\Pi_i$ is dimensionless.
    \item[6.] State the dimensionless relationships, 
    \item[7.] Simplify the relationship, if possible; this generally requires physical insight, such as experiments which determine a proportionality coefficient.
\end{enumerate}

One important application of dimensional analysis is in model testing. It is often difficult to test a full-scale prototype, both because of practical wind tunnel size restrictions and expenses in producing full-size models. Dimensional analysis allows one to design experiments where the scale model testing faithfully represents and predicts the full-scale behavior. This is the case when the model and prototype are \textbf{dynamically similar}. For a model and prototype to by dynamically similar, they must be \textbf{scale similar} (i.e. boundary conditions (geometry) and initial conditions are proportionally scaled) and their \textit{dimensionless groups must match}. So long as flow is dynamically similar, the fluid used does not matter. In fact, it is often desirable to use water when testing aerodynamics, because this matches Reynolds numbers with considerably lower speeds.

\begin{framed}
\textbf{Nondimensionalization of Governing Equations}. If a governing equation is known, it can be non-dimensionalized to determine the dimensionless parameters characterizing its flow. For example, consider the conservation of linear momentum, given by \[\rho \left(\frac{d\vec{u}}{dt} + \vec{u}\cdot \nabla\vec{u}\right) = -\nabla p+\mu\nabla^2\vec{u} + \rho\vec{g}\] Define the following dimensionless parameters: \[\vec{u}^* = \vec{u}/V,\ \ \ \vec{x}^* = \vec{x}/L,\ \ \ t^* = t/\tau,\ \ \  p^* = p/p_0,\ \ \ \vec{g}^* = \vec{g}/g,\ \ \  \nabla^* = L\nabla\] where $V,L,\tau, p_0$ are \textit{characteristic} properties of the flow. By making substitutions, linear momentum becomes nondimensionalized as follows: \[\left[\frac{L}{TV}\right]\frac{d\vec{u}^*}{dt^*} + \vec{u}^*\cdot \nabla^*\vec{u}^* = -\left[\frac{p_0}{\rho V^2}\right] \nabla^*p^* + \left[\frac{\mu}{\rho VL}\right] (\nabla^*)\vec{u}^* + \left[\frac{gL}{V^2}\right]\vec{g}^* \iff \text{St}\frac{d\vec{u}^*}{dt^*} + \vec{u}^*\cdot \nabla^*\vec{u}^* = -\text{Eu} \nabla^*p^* + \frac{1}{\text{Re}} (\nabla^*)\vec{u}^* + \frac{1}{\text{Fr}^2}\vec{g}^*\] Therefore, the flow is characterized by the Strouhal number, the Euler number, the Reynolds number, and the Froude number.
\end{framed}

\subsection{Special Cases: Internal and External Flows}

External and internal flows are special cases which often arise in engineering applications. Some general techniques are derived to simplify and evaluate simple systems. In both external and internal flows, behavior varies substantially between turbulent and laminar regimes; Reynolds number dependence is discussed. In general, a flow regime is \textbf{laminar} for $Re<2000$, and a flow regime is \textbf{turbulent} for $Re>4000$. Between these is the \textbf{transition} regime, characterized by $Re\in (2000,4000)$.

Flow bounded by walls is called \textbf{internal flow}. Common examples include flow through pipes and ducts, which is of interest for many engineering applications. Key problems here involve energy loss as fluid flows through the pipes.

\begin{framed}
\textbf{Characteristics of Laminar and Turbulent Pipe Flow}. The Reynolds number characterizing pipe flow is related to the pipe diameter, and for a circular pipe is typically given by: \[Re = \frac{\rho v D}{\mu}\] The \textit{entrance length}, $L_e$, is the distance from an entrance or geometry change after which flow is fully developed. $L_e$ varies differently for different regimes, and is given by: \[\frac{L_e}{D} \approx \begin{cases}
    0.06Re,& \text{laminar flow}\\
    4.4Re^{1/6},& \text{turbulent flow}
\end{cases}\]

When flow is fully developed, it has a defined velocity profile. In laminar flow, this profile can be derived from Navier Stokes, and is a parabolic profile with $U_\text{max}=2U_\text{avg}$.

In a turbulent pipe flow, there is a different velocity profile at each position $z$ and time $t$, and the average profile is a flatter parabola, with $U_\text{max}\approx U_\text{avg}$. The flatter parabolic profile means there is a larger velocity gradient, and therefore greater shear, at the walls. This means that, in turbulent flows, there is a large frictional drag penalty compared to laminar flows.
\end{framed}

The extended Bernoulli equation can be adapted for general pipe flows, including turbulent flows. In particular, by dividing the mechanical energy equation by $g$ and scaling kinetic energy terms by a coefficient $\alpha$ to scale for the non-uniform profiles, the mechanical energy equation becomes: \[\frac{P_2}{\rho g} + \alpha_2 \frac{v_2^2}{2g} + z_2 = \frac{P_1}{\rho g} + \alpha_1 \frac{v_1^2}{2g} + z_1 + h_S - h_L\] where $h_S$ is the head shaft, given by $h_s = \dot W / (\dot mg)$, and $h_L$ is the head loss. In general, the coefficients $\alpha$ depend on the flow regime: \[\alpha = \begin{cases}
    1, & \text{uniform flow}\\
    1.08, & \text{turbulent flow}\\
    2, & \text{laminar flow}
\end{cases}\] Head loss can be divided into major and minor losses. Major head losses are related to pressure drops which occur due to wall shear stress; minor head losses are related to pipe components, like bends and nozzles. Major head losses are given by: \[h_{L,\text{major}} = \frac{\Delta P}{\rho g} =  f \cdot \frac{L}{D} \cdot \frac{v^2}{2g}\] where $f$ is the \textbf{friction factor}; for laminar flow, $f$ is only a function of $Re$, and for turbulent flow, wall friction dominates and $f$ is only a function of the roughness $\epsilon/D$, and $f$ is only an empirical relationship; $f$ can be computed or read from a table, such as a \textbf{Moody chart}, according to the following criterion: \[f = \begin{cases}
    64/Re, & \text{laminar flow}\\
    f(\epsilon/D), & \text{turbulent flow}
\end{cases}\] where $f(\varepsilon/D)$ is the experimental relationship as a function of roughness. The minor losses are computed as follows: \[h_{L,\text{minor}} = K_L \frac{v^2}{2g}\] where $K_L$ is a function of geometry. Total head loss can be found by summing all the major and minor head losses in a section of pipe. 

This analysis can be extended from circular pipes to approximate flow in any arbitrary non-circular duct. In particular, the \textit{hydraulic diameter} $D_h$ can be introduced instead of circular diameter $D$, where: \[D_h = \frac{4A}{P}\] where $A$ is the cross-sectional area of the duct, and $P$ is the perimeter of the duct. Using $D_h$ as the characteristic length in head loss calculations gives an approximation for non-circular geometries.

Flow over a body immersed in a fluid is called \textbf{external flow}. Flow past any object with a solid surface develops a thin \textit{boundary layer} in which viscous effects are important, even if the flow far from the body is inviscid.

Consider a flat plate of infinite length, with uniform free-stream velocity $U$ parallel to the plate. Let $x$ be the direction of flow. As flow proceeds over the plate, fluid near the wall becomes stationary, resulting in a gradient from zero-velocity at the wall to the free-stream velocity $U$ at some thickness $\delta$ from the wall. This thickness, $\delta$, is known as the \textit{boundary layer}.

Over sufficient length, the boundary layer tends to transition from a \textit{laminar boundary layer} to a \textit{turbulent boundary layer}. Note that, by choosing $x$ as the characteristic length, the Reynolds number characterizing flow over the plate is given by: \[Re_x = \frac{\rho U x}{\mu}\] The \textit{boundary layer} is laminar for $Re_x<2\times 10^5$, and is turbulent for $Re_x > 3\times 10^6$.

\begin{framed}
\textbf{Boundary Layer and Resulting Wall Shear}. There are several metrics for thickness which relate some quantities: 
\begin{enumerate}
    \item[] \textbf{Boundary Layer Thickness}. The boundary layer thickness, also called he \textit{disturbance thickness}, $\delta_{99}$, is the thickness at which 99$\%$ of the free-stream velocity is recovered: \[\delta_{99} = \displaystyle y\rvert_{u=0.99U}\]
    \item[] \textbf{Displacement Thickness}. The \textit{displacement thickness}, $\delta^*$, is the displacement from the wall which results in an equivalent mass flow deficit. Displacement thickness is given by: \[\delta^* = \int_0^\infty \left(1-\frac{u}{U}\right)dy\]
    \item[] \textbf{Momentum Thickness}. The \textit{momentum thickness}, $\theta$, is the displacement from the wall which results in an equivalent momentum deficit. Momentum thickness is given by: \[\theta = \int_0^\infty \frac{u}{U}\left(1-\frac{u}{U}\right)dy\]
\end{enumerate}

Under the assumption of steady, two-dimensional flow with a thin boundary layer, the following relationship for wall shear stress $\tau$ can be derived: \[\frac{\tau}{\rho} = \frac{d}{dx}(U^2\theta) + \delta^* U \frac{dU}{dx}\tag{Momentum Integral Equation}\] Note that $U$ is from the outer flow; if the outer flow is inviscid, relations can be found with Bernoulli's equation. In particular, by differentiating Bernoulli's equation: \[P + \frac{1}{2}\rho U^2 = c \implies \frac{dP}{dx} = -\rho U \frac{dU}{dx}\]
\end{framed}

For many geometries and flow conditions, a boundary layer may depart from the surface, resulting in a region of reversed flow near the surface; this phenomenon is called flow \textbf{ separation}. A body is called \textit{bluff}, or \textit{blunt}, if large flow separations occur (i.e. a sphere of a cube), and a body is called \textit{streamlined} if flow remains mostly attached (i.e. a low-AoA airfoil). Quantitatively, flow separation occurs at a point on the surface where the velocity gradient cannot overcome the pressure difference over the body, meaning $\displaystyle\frac{\partial u}{\partial y}\bigg\rvert_{(x,0)}=0$. Past this critical point, the velocity gradient reverses due to the pressure gradient.

Laminar boundary layers are generally more prone to separation than a turbulent boundary layer. Separation of a laminar boundary layer generally leads to transition flow, which energizes the flow and leads to reattachment further along the geometry, resulting in a separation bubble.

Force in the direction of the fluid stream in known as \textbf{drag}; flow opposite the direction of the fluid stream is known as \textbf{thrust}. Force perpendicular to the free stream is known as \textbf{lift} or \textbf{downforce}, depending on the direction. Dimensional analysis can be used to derive nondimensional relationships for lift and drag, written by convention as the \textit{drag coefficient}, $C_D$, and the lift coefficient, $C_L$: \[C_D = \frac{F_D}{\frac{1}{2}\rho v^2 A},\hspace{0.5in}C_L = \frac{F_L}{\frac{1}{2}\rho v^2 A}\] In both coefficients, $A$ is any characteristic area, which can result in different coefficients. Therefore, by convention, for a bluff body, $A$ is taken to be the cross-sectional area affected by the fluid, and for an airfoil the area is taken to be the chord length times span length, $A=cs$. Note that the quantities $C_LA$ and $C_DA$ are constant for a body at given conditions, regardless of the choice of characteristic area.