\section{Dynamics and Vibrations}

\begin{center}
    Notes from \textbf{MECHENG 240: Introduction to Dynamics and Vibrations}

    Taken at University of Michigan, Winter 2025

    using \textit{Dynamics}, by Meriam and Kraige
\end{center}

\subsection{Particle Kinematics} %%% 2.1 -- 2.7

Kinematics is the description of motion without reference to forces acting on an object. Particle kinematics is concerned with the movement of bodies which are physically small compared to their paths. 

For a particle a distance $s(t)$ from a fixed point in direction of unit vector $\hat e$, position may be defined by \[\vec{r}(t) = s(t)\hat e\] Differentiation gives the general relationship for velocity: \[\vec{v}(t) = \vec{r}'(t) = s'(t)\hat e + s(t)\frac{d\hat e}{dt}\] In the case of \textit{rectilinear motion} (motion along a line), $\hat{e}$ is constant with respect to time and thus its derivative is zero. Further, there is only one vector direction to consider, so rectilinear motion may be generally reduced to scalar quantities with a positive and negative direction, giving the following relationships: \[v = \frac{ds}{dt} = \dot s \hspace{0.5in} a = \frac{dv}{dt} = \dot v\hspace{0.5in} a=\frac{d^2s}{dt^2} = \ddot s\] Additionally, in the case of rectilinear motion, time may be eliminated by applying the chain rule: \[a = v\frac{dv}{ds} \iff vdv = ads\] For the special case of \textit{constant acceleration}, integration of these equations always yields the following direct relationships between displacement, velocity, and acceleration:
\begin{shaded}
    \textbf{Uniformly Accelerated Motion}. For a particle experiencing \textit{constant acceleration}, the following relationships hold:
    \[v = v_0 + at\]
    \[v^2 = v_0^2 + 2a\Delta s\]
    \[s = s_0 + v_0t + \frac{1}{2}at^2\]
\end{shaded}

For motion in higher dimensions (i.e. in a plane or a space), the differential relationship for vectors continues to hold. In particular, for a particle a distance $r$ from the origin, \[\vec{v} = \frac{d\vec{r}}{dt}\hspace{0.5in}\vec{a}=\frac{d\vec{v}}{dt}\] 

\newpage

Most dynamics problems may be greatly simplified when approached in an appropriate coordinate system. Three useful \textit{orthonormal coordinate systems} are outlined:

\vspace{-2em}
\begin{enumerate} %%% need to fill these in
    \item[] \textbf{Rectangular coordinates}. Uses fixed unit vectors in each coordinate direction, generally $\hat i$, $\hat j$, $\hat k$. Requires a fixed (non-inertial) reference point for an origin. Position is given by \[\vec{r} = x\hat i + y\hat j + z\hat k\] By differentiation, velocity is given by \[\vec{v} = \dot x \hat i + \dot y \hat j + \dot z \hat k\] And by differentiation again, acceleration is given by \[\vec{a} = \ddot x \hat i + \ddot y \hat j + \ddot z \hat k\]
    \item[] \textbf{Normal and Tangential coordinates}. Uses two moving vectors to describe motion of a particle: $\hat e_t$ tangential to travel, and $\hat e_n$ in the direction of the circle which best fits the path. Generally most applicable when the problem may be simplified at each instant to a planar problem. If necessary, a basis vector $\hat k$ may be defined perpendicular to both $\hat e_t$ and $\hat e_n$. Velocity is given by \[\vec{v}(t) = v\hat e_t\] and acceleration is given by \[\vec{a}(t) = \dot v \hat e_t + \frac{v^2}{\rho}\hat e_n\] where $\rho$ is the radius of curvature of the path at that moment in time.
    \item[] \textbf{Polar and Cylindrical coordinates}. Uses a unit vector $\hat e_r$ pointing radially from the origin, a vector $\hat e_\theta$ pointing in the direction of rotation, and a fixed vector $\hat k$ about which the particle rotates. In two dimensions, $\hat k$ is always zero and thus can be ignored. Position is given by \[\vec{r} = r\hat e_r + z\hat k\] By differentiation, velocity is given by \[\vec{v} = \dot r \hat e_r + r\dot\theta \hat e_\theta + \dot z \hat k\] And by differentiation again, acceleration is given by \[\vec{a} = (\ddot r - r\dot\theta^2)\hat e_r +(r\ddot\theta + 2 \dot r \dot \theta)\hat e_\theta + \ddot z \hat k\]
\end{enumerate}
The vectors $\vec{r}$, $\vec{v}$, and $\vec{a}$, and any other vectors, are \textit{equivalent} in any coordinate system. By using a change of basis, any basis vector for a particular coordinate system may be decomposed into components and converted into another coordinate system, i.e. by a rotation matrix or trigonometric analysis.


%%% this is where i stopped tbh


\subsection{Particle Kinetics} %%% 3.1 -- 3.9, 3.12

Kinetics is the study of motion caused by unbalanced forces. Newton's Second Law is a description of motion (acceleration), weighted by a property of the object in motion (inertia; or, mass): \[\vec{F} = m\vec{a}\hspace{0.5in}\text{(Newton's 2nd Law)}\] For a body acted upon by several forces, the vector sum of the forces acting on an object gives the net force and thus net acceleration: \[\sum \vec{F} = \vec{F}_\text{net} = m\vec{a}\] Some common forces are as follows:

Combined with a kinematic analysis, kinetics relates the forces acting on an object to its path of travel, where either one may be found from the other. For Newton's 2nd Law to apply, the frame of reference \textit{must be inertial (non-accelerating, and/or stationary)}.

\newpage

For some analyses, it may be convenient to consider only the initial and final states of some object. This analysis is thus time-independent, and may be carried out with an energy analysis. In particular, integrating Newton's 2nd Law with respect to position gives a useful result:

\begin{eqnarray*}
    \underbrace{\int_{\vec{R_1}}^{\vec{R_2}} \vec{F}\cdot d\vec{R}}_{\text{Work, }W_{1-2}} &=& \int_{\vec{R}_1}^{\vec{R}_2} m\vec{a}\cdot d\vec{R}\\
    &=& \int_{\vec{R}_1}^{\vec{R}_2} m(\dot v \hat e_t + \frac{v_2}{\rho}\hat e_n)\cdot ds\hat e_t\\
    &=& m\int_{\vec{R}_1}^{\vec{R}_2}\dot v \hat e_t \cdot ds \hat e_t\\
    &=& m\int_{s_1}^{s_2} a ds\\
    &=& m\int_{v_1}^{v_2} vdv\\
    &=& \underbrace{\frac{1}{2}mv_2^2 - \frac{1}{2}mv_1^2}_{\text{Kinetic energy}}
\end{eqnarray*}

This result is known as the \textbf{work-energy theorem}:
\begin{shaded}
    \textbf{Work-Energy Theorem}. The net work on an object, defined by \[U_{1-2}:=\int_{\vec{R}_1}^{\vec{R}_2} \vec{F}\cdot d\vec{R}\] is equal to the difference in kinetic energy between positions $\vec{R}_1$ and $\vec{R}_2$: \[U_{1-2} = \frac{1}{2}mv_2^2 - \frac{1}{2}mv_1^2\]
\end{shaded}

Often, $d\vec{R}$ is defined with respect to a Cartesian coordinate system, thus giving $d\vec{R} = dx\hat i + dy\hat j + dz\hat k$, but $d\vec{R}$ may also be defined in another coordinate system, such as polar-cylindrical.

Energy analysis is often simplified when the forces acting on a body are \textbf{conservative forces}. A conservative force is any force where work done on a body depends only on the initial and final states and is \textit{independent of the path of travel}, as is the case with spring and gravitational forces. If a force $\vec{F}$ is conservative, it may be defined to have a \textit{potential energy}, where the work done by said force is defined as $U_F = -(V_2 - V_1)$. Potential energies for common forces are outlined:

\begin{enumerate}
    \item \textbf{Gravitational Potential Energy}. For a gravitational force $\vec{F}_g=mg$ at the Earth's surface, potential energy is given by \[V = mgz\]
    \item \textbf{Spring Potential Energy}. For a Hookean spring force $\vec{F}_s = -k\Delta s$, potential energy is given by \[V = \frac{1}{2}ks^2\]
\end{enumerate}

Given this, the work-energy theorem may be broken into work by conservative and nonconservative forces: \[\underbrace{-(V_2-V_1)}_{U_\text{conservative}} + U_\text{nonconservative} = T_2-T_1 \iff U_\text{nonconservative} = \frac{1}{2}mv_2^2 - \frac{1}{2}mv_1^2 + V_2 - V_1\]

\textbf{Momentum methods} also involve analysis of initial and final states, in this case integrating with respect to time. In particular, \begin{eqnarray*}
    \vec{F} &=& m\frac{d\vec{v}}{dt}\\
    \implies \underbrace{\int_{t_1}^{t_2} \vec{F}dt}_\text{impulse} &=& \int_{\vec{v}_1}^{\vec{v}_2}md\vec{v}\\
    &=& \underbrace{m\vec{v}_2 - m\vec{v}_1}_\text{linear momentum}
\end{eqnarray*}

As opposed to work, $m\vec{v}$ is a vector quantity which relates to velocity rather than speed.

If a system of two particles is considered in a momentum analysis, then any forces exerted from particle $A$ onto particle $B$ and vice versa are internal, and thus cancel in integration. This makes momentum analysis particularly suited to collision problems, as internal forces at the moment of collision can be neglected. A collision between two particles can be further simplified if the collision is assumed to take place over an instant. Given this, consider two particles $A$ and $B$ which undergo a collision. Their momentum balance is given by: \begin{eqnarray*}
    \cancelto{0\text{ as }\Delta t\to 0}{\int_0^{\Delta t} (\vec{F}_A + \vec{F}_B)dt} &=& \underbrace{(m_A\vec{v}_A' +m_B\vec{v}_B')}_\text{after collision} - \underbrace{(m_A\vec{v}_A + m_B\vec{v}_B)}_\text{before collision}\\
    \implies (m_A\vec{v}_A' +m_B\vec{v}_B') &=& (m_A\vec{v}_A + m_B\vec{v}_B)
\end{eqnarray*}

If a collision is a \textit{direct central impact}, the particles collide in a line, and their motion is rectilinear. This simplifies the momentum balance to one dimension:

\[m_Av_A + m_Bv_B = m_Av_A' + m_Bv_B'\]

Note that, even in this simple case, there are generally at least two unknown values--the velocities of each body after collision. In this case, a known \textit{coefficient of restitution} may be used to relate speeds along the axis of impact. The coefficient of restitution for two bodies is given by \[e = \frac{v_B'-v_A'}{v_A-v_B},\hspace{0.25in}0\leq e\leq 1\]
If $e=0$, the objects move at the same final speed and therefore stick together in a \textit{perfectly inelastic collision}. If $e=1$, then one body transfers all its energy to the second body in a \textit{perfectly elastic collision}.

For an \textit{oblique impact}, objects make contact along a normal axis. Velocity in the normal direction changes according to the coefficient of restitution, while velocity along the tangential axis is not changed during collision.

\newpage

\subsection{Vibration of Particles} % 8.1-8.2

Many real-world systems may be modeled as particles undergoing vibration--in particular, as spring-mass systems.

A particle undergoing vibrations is either \textit{free}, with no external forces besides the restoring force causing oscillations, or \textit{forced}, with some kind of external force acting on the system.

\textbf{Undamped, Free Vibrations}

In the absence of external forces, such a system is said to undergo \textit{free vibrations}. Typical applications involve a Hookean spring, such that the restoring force caused by the spring varies linearly with displacement. The standard equation of motion for such a system is as follows:

\begin{eqnarray*}
    m\ddot x &=& -kx\\
    m\ddot x + kx &=& 0\\
    \ddot x + \frac{k}{m}x &=& 0\\
    \text{let }\omega_n^2 = \frac{k}{m} &\longrightarrow& \boxed{\ddot x + \omega_n^2 x = 0}
\end{eqnarray*}

Here, $\omega_n$ is the \textit{undamped natural frequency} for the vibrations. Using standard methods (i.e. characteristic equations), this differential equation can be shown to have the following solution:

\[\ddot x + \omega_n^2x = 0 \implies \boxed{x(t) = x_0\cos(\omega_nt) + \frac{\dot x_0}{\omega_n}\sin(\omega_n t)}\]

This may alternatively be written as a single sinusoidal function: \[x(t) = C\cos(\omega_nt +\phi),\hspace{0.5in}C:=\sqrt{x_0^2+\left(\frac{\dot x_0}{\omega_n}\right)^2},\hspace{0.2in}\phi :=\arctan\left(\frac{\dot x_0}{\omega_nx_0}\right)\]

\textbf{Damped, Free Vibrations}

In many applications, a spring is paired with a \textit{damper}, which contains some fluid and thus produces a force proportional to velocity which resists motion. The standard equation of motion for such a system is as follows:

\begin{eqnarray*}
    m\ddot x &=& -c\dot x -kx\\
    m\ddot x + c\dot x+ kx &=& 0\\
    \ddot x + \frac{c}{m}\dot x + \frac{k}{m}x &=& 0\\
    \text{let }\zeta = \frac{c}{2m\omega_n} &\longrightarrow& \boxed{\ddot x + 2\zeta\omega_n \dot x + \omega_n^2x = 0}
\end{eqnarray*}

\newpage

Here, $\zeta$ is the \textit{damping ratio} for the system. If a damper is present in a system, $\zeta > 0$. Using method of characteristics, $\lambda_{1,2} = -\zeta\omega_n \pm \omega_n\sqrt{\zeta^2-1}$, and the form of the solution to the differential equation depends on $\zeta$:

\begin{enumerate}
    \item[] \textbf{Underdamped}, $\zeta < 1$: system oscillates and gradually decays to equilibrium. Frequency is reduced to $\omega_d = \omega_n \sqrt{1-\zeta^2}$: \[x(t) = e^{-\zeta\omega_nt}\left(x_0\cos(\omega_d t)+\frac{\dot x_0+\zeta\omega_nx_0}{\omega_d}\sin(\omega_d t)\right)\]
    This equation is also valid for systems without a damper, where $\zeta=0$.
    \item[] \textbf{Critically damped}, $\zeta = 1$: system returns to equilibrium as fast as possible. Solution is of the form \[x(t) = A_1e^{\lambda t} + A_2te^{\lambda t}\] where $A_1$ and $A_2$ are found by solving for initial conditions.
    \item[] \textbf{Overdamped}, $\zeta > 1$: system decays exponentially to equilibrium, but at a slower rate than critical damping. Solution is of the form \[x(t) = A_1e^{\lambda_1 t}+A_2e^{\lambda_2 t}\]
\end{enumerate}

The \textit{time constant} for a system is a measure of a system's decay rate. In general, for a decaying exponential $x_0e^{-at}$, the time constant is defined as the time $\tau$ by which $x_0 e^{-a\tau} = e^{-1}$, so $\tau = \frac{1}{a}$. For the systems under consideration here, the time constant is given by \[\tau = \frac{1}{\zeta w_n}\] At a time $4\tau$, a system has decayed by $e^{-4}\approx 98\%$, at which point a system may generally be approximated as having returned to rest.

\subsection{Rigid Body Kinematics} %%% 5.1 -- 5.2, 5.4 -- 5.7

A \textbf{rigid body} is a collection of particles for which the distance between particles does not change or may otherwise be approximated as constant. A body may move with both \textit{translation} and \textit{rotation}. For translational motion, a body moves along its center of mass, and thus relationships from particle kinematics apply directly. New kinematic quantities are required to describe the rotation of a body.

Similar to particle kinematics, angular velocity and acceleration are given by a differential equation on angle $\theta$:

\[\omega = \frac{d\theta}{dt} = \dot \theta \hspace{0.5in} \alpha = \frac{d\omega}{dt} = \dot \omega\hspace{0.5in} \alpha=\frac{d^2\theta}{dt^2} = \ddot \theta\] Again, time may be eliminated, yielding a differential equation in terms of differential quantities: \[\omega d\omega = \alpha d\theta\]

For a general rigid body, given two points $o$ and $p$ on a rigid body, absolute motion is given by the following relationship:

\begin{shaded}
General kinematics of a rotating body. Given two points $o$ and $p$ on a rigid body, where the distance from $o$ in the direction of $p$ is given by $\vec{r}_{p/o}$, absolute velocities and accelerations for the body are given by:
    \[\vec{v}_p = \vec{v}_o + \vec{\omega}\times \vec{r}_{p/o}\]
    \[\vec{a}_p = \vec{a}_o + \vec{\alpha} \times \vec{r}_{p/o} + \vec{\omega}\times (\vec{\omega} \times \vec{r}_{p/o})\]
\end{shaded}

An important application is \textit{rolling without slipping}. For a wheel which rolls without slipping, the velocity as a point on the exterior rim of the wheel approaches the contacting surface shrinks to zero and is therefore momentarily at rest, experiencing \textit{static friction}. A velocity or acceleration analysis shows that \[v = -\omega r,\hspace{0.5in}a=-\alpha r\]

It is also possible to analyze motion between two separate rigid bodies directly:

\begin{shaded}
Kinematics between two rotating bodies. Given a point $o$ on one body and a point $p$ on another (or the same) body, the motion of point $p$ in terms of $o$ is given by:
    \[\vec{v}_p = \vec{v}_o + \vec{\omega}\times\vec{r}_{p/o} + \vec{v}_{p\text{,rel }o}\]
    \[\vec{a}_p = \vec{a}_o + \vec{\alpha}\times\vec{r}_{p/o}+\vec\omega\times(\vec{\omega}\times\vec{r}_{p/o}) + 2\vec\omega \times \vec{v}_{p\text{,rel }o} + \vec{a}_{p\text{,rel }o}\]
\end{shaded}

\subsection{Rigid Body Kinetics}

Most generally, a rigid body has six degrees of freedom: three positional, and three rotational. Newton's equation describes positional acceleration when applied to the center of mass, and Euler's equation describes rotation.

Center of mass $\vec{r}_c$ of an object may be computed for a rigid body comprised of either a continuous or discrete distribution of points: \[\vec{r}_c = \frac{m_1\vec{r}_1 + m_2\vec{r}_2+\dots+m_n\vec{r}_n}{m_1+m_2+\dots+m_n} = \int_0^M \frac{\vec{r}dm}{M}\]

Euler's equation relates the moments on a body about an axis to the body's angular acceleration. Moments may be computed as the cross product between the distance to a force and the force itself: \[\vec{M}_c = \vec{r}_{p/c} \times \vec{F}_p\]
For planar motion, moments must act in or out of the plane, so a single Euler's equation describes motion of a body as there is only one axis of rotation. Euler's equation may be written equivalently in several ways:

\begin{shaded}
    \textbf{Euler's equation}. Euler's equation may be equivalently expressed as:
    \begin{enumerate}
        \item Euler's equation about the center of mass, $c$: \[\sum \vec{M}_c = I_c\vec{\alpha}\]
        \item Euler's equation about a fixed point, $o$: \[\sum \vec{M}_o = I_o\vec{\alpha}\]
        \item Euler's equation about an arbitrary point $p$: \[\sum \vec{M}_p = I_p\vec{\alpha} + \vec{r}_{c/p} \times m\vec{a}_p\]
    \end{enumerate}
\end{shaded}
The \textit{mass moment of inertia} about the center of mass of a rigid body is computed by an integral:
\[I_c = \int_m r^2dm\]
The \textit{parallel axis theorem} allows the calculation of a moment of inertia about any point on a body if the moment of inertia about the center of mass $c$ is known. In particular, the moment of inertia about a point $p$ on a body is given by \[I_p = I_c + md^2\] where $d^2$ is the scalar distance between $c$ and $p$. Therefore, $I_c$ is necessarily the smallest moment of inertia for a body.

For a rigid body, the kinetic energy may be expressed as a combination of the energy caused by linear and rotational velocity: \[T = \frac{1}{2}mv_c^2 + \frac{1}{2}I_c\omega^2 = \frac{1}{2}I_p\omega^2\text{ if }\vec{v}_p = \vec{0}\]
Potential energy is calculated at the center of mass of a rigid body. With this, the \textit{work-energy theorem} derived for particles may be applied to rigid bodies.

\subsection{Vibration of Rigid Bodies}

Rigid body vibrations may take advantage of both Newton's and Euler's equations to describe the motion of a body.

\textbf{Harmonically Forced Vibrations}

The system under consideration is a mass attached to a spring of constant $k$, a damper with damping ratio $c$, and a forcing term $F(t)$. The equation of motion is therefore \begin{eqnarray*}
    m\ddot x &=& -kx -c\dot x +F(t)\\
    \implies \ddot x + \frac{c}{m}\dot x + \frac{k}{m}x &=& \frac{F(t)}{m}\\
    \implies \ddot x + \frac{c}{m}\dot x + \frac{k}{m} x &=& \frac{F_0}{m}\cos(\omega t)\text{ for forcing term with frequency } \omega
\end{eqnarray*}

The solution to this differential equation is a sum of the homogeneous and particular solution $x_c(t) + x_p(t)$. The homogeneous solution has already been found; the particular solution may be found by making a guess of the same form as the forcing term.
\begin{eqnarray*}
    x_p(t) &=& A_1\cos(\omega t)+A_2\sin(\omega t)\\
    &=& X\cos(\phi)\cos(\omega t)+X\sin(\phi)\sin(\omega t)\\
    &=& X\cos(\omega t + \phi),\text{ where }\phi=\arctan(-A_2/A_1)\text{ and } X=\sqrt{A_1^2+A_2^2}
\end{eqnarray*}
Solving the differential equation by standard methods, this gives the following: \[X = \frac{F_0/m}{\sqrt{(2\zeta \omega_n\omega)^2+(\omega_n^2-\omega^2)^2}}\text{ and } \phi = \arctan\left(\frac{-2\zeta\omega_n\omega}{\omega_n^2-\omega^2}\right)\]
Defining the tuning ratio $\eta = \omega/\omega_n$, these constants may be alternatively defined in terms of the ratio of forcing frequency to natural frequency of the undamped system:
\[X=\frac{F_0/k}{\sqrt{(2\zeta\eta)^2+(1-\eta^2)^2}}\text{ and }\phi = \arctan\left(\frac{-2\zeta\eta}{1-\eta^2}\right)\] The \textit{dynamic magnification factor} $M$ is a measure of amplitude under forcing and can be used to determine resonance (i.e. as $M\to \infty$). $M$ is defined as \[M:= X/(F_0/k) = \frac{1}{\sqrt{(2\zeta\eta)^2+(1-\eta^2)^2}}\]

\textbf{Base Excitations}

If the base on which a spring-mass system is attached is free to move in a known manner, the spring-mass system has a different equation of motion with a forcing term from the base:
\[m\ddot x + c(\dot x+\dot z) + k(x+z) = 0 \iff m\ddot x + c\dot x + kx = \underbrace{c\dot z + kz}_\text{forcing term}\]

Given a periodic $z(t)$ as input, this becomes a harmonic forcing term. In this case, the magnification factor is given by the ratio \[M = \frac{X}{z_0} = \sqrt{\frac{1+(2\zeta\eta)^2}{(2\zeta\eta)^2+(1-\eta^2)^2}}\]